\input alist.tex
\hoffset-.54cm
\voffset-.54cm
\hsize17cm
\vsize25.07cm
\def\\{\char92\relax}\def\{{\char123\relax}\def\}{\char125\relax}\def\#{\char35\relax}
\noindent This document tries to show how the package alist.tex may be used
\bigskip
\noindent copy
\smallskip
The control sequence {\tt\\copy} copies the {\it definition} of the {\it control sequence} in {\tt\#2} and changes the {\it definition} of the {\it control sequence} in {\tt\#1}.\hfil\break
{\tt\\def\\a\{a\}\hfil\break
\\def\\b\{b\}\hfil\break
\\copy\\a\\b\hfil\break
\\a} $\rightarrow$ b\hfil\break
Note that  since the {\it definition} is actually copied it stays the same even if the {\it control sequence} it was copied from is redefined.\hfil\break
{\tt\\def\\b\{c\}\hfil\break
\\a} $\rightarrow$ b\hfil\break
In contrast if you were to just redefine {\tt\\a} as {\tt\\b} its value would be bound to the {\it definition} of {\tt\\b}\hfil\break
{\tt\\def\\a\{\\b\}\hfil\break
\\a\hbox{ $\rightarrow$ c}\hfil\break
\\def\\b\{d\}\hfil\break
\\a} $\rightarrow$ d\hfil\break
You can also copy a {\it control sequence}'s {\it definition} into itself.\hfil\break
{\tt\\copy\\a\\a\hfil\break
\\a} $\rightarrow$ d
\medskip
\noindent append
\smallskip
The control sequence {\tt\\append} appends the{\it definition} of the {\it control sequence} in {\tt\#2} to the end of the {\it definition} of the {\it control sequence} in {\tt\#1}.\hfil\break
{\tt\\def\\a\{a\}\hfil\break
\\def\\b\{b\}\hfil\break
\\append\\a\\b\hfil\break
\\a} $\rightarrow$ ab\hfil\break
You can also append a {\it control sequence}'s {\it definition} to itself.\hfil\break
{\tt\\append\\a\\a\hfil\break
\\a} $\rightarrow$ abab
\medskip
\noindent contextwidth
\smallskip
The {\it control sequence} {\tt\\contextwidth} takes a {\tt\\dimen} register in{\tt\#1} and anything in{\tt\#2}. It then stores the width {\tt\#2} would have, were it written next to the arguments this controlsequence previously accepted in {\tt\#2}.\hfil\break
{\tt\\newdimen\\dimena\hfil\break
\\newdimen\\dimenb\hfil\break
\\contextwidth\{\\dimena\}\{\\advance\\dimenb10pt\}\hfil\break
\\the\\dimena\hbox{ $\rightarrow$ 0pt}\hfil\break
\\contextwidth\{\\dimena\}\{\\vrule width\\dimenb\}\hfil\break
\\the\\dimena} $\rightarrow$ 10pt
\bigskip
\noindent list
\smallskip
The {\it control sequence} {\tt\\list} provides an ajustable method of typesetting lists it's argument {\tt\#1} defines the bullet's to be used for the content given in {\tt\#2}. The simplest form of list only uses {\tt\\item}'s. These are automatically aligned behind the specified bullets.
{\tt\\list\{-\\ \}\{\hfil\break
\\item\{This is the content of the first item.\}\hfil\break
\\item\{This is the content of the second item.\hfil\break
Items may consist of multiple rows.\hfil\break
For a text of sufficient length I had to strain my imagination.\}\}\hfil\break}
The previous code produces the following list:\hfil\break
\list{-\ }{
\item{This is the content of the first item.}
\item{This is the content of the second item. Items may consist of multiple rows. For a text of sufficient length I had to strain my imagination.}}
\noindent{\tt\\bullet} is defined such that bullets may be influenced by previous ones in the same list. Bullets may also be redefined within a list. This can happen either in bullet's or inbetween items. The content of items are encapsulated and can therefore not influence {\tt\\bullet}.\hfil\break
{\tt\\newcount\\bulletcount\hfil\break
\\list\{\\advance\\bulletcount1 \\number\\bulletcount)\\ \}\{\hfil\break
\\item\{Using this method one can number items.\hfil\break
This item is labeled the first.\}\hfil\break
\\item\{This item is labeled second.\hfil\break
The number convieniently changes without manual any alteration.\}\hfil\break
\{\\def\\bullet\{\$\\bullet\$\\ \}\hfil\break
\\item\{Bullets may also be redefined inbetween any two items.\}\}\hfil\break
\\item\{When counting\}\hfil\break
\\item\{high enough\}\hfil\break
\\item\{you will see,\}\hfil\break
\\item\{that\}\hfil\break
\\item\{the list\}\hfil\break
\\item\{automatically\}\hfil\break
\\item\{choses\}\hfil\break
\\item\{the appropriate spacing.\}\}}\hfil\break
The previous code produces the following list:\hfil\break
\newcount\bulletcount
\list{\advance\bulletcount1 \number\bulletcount)\ }{
\item{Using this method one can number items.
This item is labeled the first.}
\item{This item is labeled second.
It's number conviniently increased without extra work.}
{\def\lbullet{$\bullet$\ }
\item{Bullets may also be redefined inbetween any two items.}}
\item{When counting}
\item{high enoug}
\item{you will see,}
\item{that}
\item{the list}
\item{automatically}
\item{choses}
\item{the appropriate spacing.}}
\noindent One can't just redefine {\tt\\bullet}, inbetween {\tt\\item}'s but also put regular text.\hfil\break
{\tt\\list\{-\\ \}\{\hfil\break
\\item\{The text below this item is not an item itself.\}\hfil\break
\\noindent This is not an item\\hfil\\break\hfil\break
\\item\{The text above this item is not an item itself.\}\hfil\break
\}}\hfil\break
The previous code produces the following list:\hfil\break
\list{-\ }{
\item{The text below this item is not an item itself.}
\noindent This is not an item.\hfil\break
\item{The Text above this item is not an item itself.}}
Lists can also be nested in other lists.\hfil\break
{\tt\\list\{\$\\bullet\$\\ \}\{\hfil\break
\\item\{\\list\{-\\ \}\{\par
\hskip.5\parindent\\item\{Some text.\}\par
\hskip.5\parindent\\item\{\\dots\}\hfil\break
\}\}\hfil\break
\\item\{Second item.\}\}}\hfil\break
The previous code produces the following list:\hfil\break
\list{$\bullet$\ }{
\item{\list{-\ }{
  \item{Some text.}
  \item{\dots}
}}
\item{Second item.}}
\bye

