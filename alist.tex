% Script: list.tex
% Purpose: Defines a custom list environment (\list) that allows for flexible formatting of list elements, including the option to automatically number them.
% Author: [ukofk]

% Define \newdimen commands outside of \list to avoid redefinition
\newdimen\bulletwidth % Define dimension to measure the width of bullet points
\newdimen\bulletswidth % Define dimension to store the maximum bullet width

% Define \list environment
\def\list#1#2{%
  % Define \append macro to append to the control sequence
  \def\append##1##2{%
    \expandafter\def\expandafter##1\expandafter{##1##2}%
  }%

  % Define \curcon to store the concatenated list of all arguments
  \def\curcon{}%

  % Define \precon to store the concatenated list of all but the last argument
  \def\precon{}%

  % Define \conlen to measure the width of the current argument and update bulletwidth
  \def\conlen##1{%
    % Store the previous context in \precon
    \expandafter\def\expandafter\precon\expandafter{\curcon}%
    % Append the current argument to \curcon
    \append\curcon{##1}%
    % Measure the width of \curcon
    \setbox0=\hbox{\curcon}%
    \bulletwidth=\wd0
    % Measure the width of \precon and subtract from \bulletwidth
    \setbox0=\hbox{\precon}%
    \advance\bulletwidth by -\wd0
  }%

  % Calculate maximum bullet width
  {\def\item##1{%
    \conlen{#1}%
    % Update \bulletswidth if current bullet width is greater
    \ifdim\bulletwidth>\bulletswidth%
      \bulletswidth\bulletwidth%
    \fi%
  }%
  #2 % Print the content of the list
  % Redefine \item with indentation
  {\def\item##1{%
    \noindent#1\vtop{\vbox{\advance\hsize-\bulletswidth\noindent##1}}\par
  }#2}}
}

%%Set initial value for list counter
 %\newcount\listcount

%%Example usage of \list with automatic numbering
 %\list{\advance\listcount by 1 \number\listcount.\ }{%
 %  \item{First item}
 %  \item{Second item with longer text}
 %  \item{Third item}
 %}

 %\bye
