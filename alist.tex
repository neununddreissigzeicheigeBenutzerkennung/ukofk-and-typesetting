%%% This package aims to provide a versitile \list controll sequence which takes custom \lbullet's (#1) and then takes care of spacing for the \item's in #2. For more detailled information view the example document.
%%% A  few extra controll sequences where used to achive \list's functionality which if not helpfull may at least be interesting to some and are thus also shown off in the example document.

% This dimension is for keeping track of individual bullet's width.
\newdimen\bulletwidth%
% This dimension is for keeping track of the maximum bullet width in a list.
\newdimen\bulletswidth%
% This dimension is used for transcending the borders of one \box whithout using \global.
\newdimen\transcendebullets%
% This controll sequence is for keeping track of the arguments #2 of contextwidth up to the current one.
\def\currentcontext{}%
% This controll sequence is for keeping track of the arguments #2 of contextwidth up to the previous one.
\def\previouscontext{}%

%% This controll sequence copys the content of the controll sequence #2 to the definition of the controll sequence #1.
\def\copy#1#2{%
  \expandafter\def\expandafter#1\expandafter{#2}%
}%
%% This controll sequence apeends the content of the controll sequence #2 to the definition of the controll sequence #1.
\def\append#1#2{%
  \expandafter\expandafter\expandafter\def%
  \expandafter\expandafter\expandafter#1%
  \expandafter\expandafter\expandafter{%
    \expandafter#1#2%
  }%
}
%% This controll sequence coputes the length of #2 in context with it's previous values and store it in dimen #1.
\def\contextwidth#1#2{%
  \copy\previouscontext\currentcontext%
  \append\currentcontext{#2}%
  \setbox0\hbox{\currentcontext}%
  \immediate#1\wd0%
  \setbox0\hbox{\previouscontext}%
  \immediate\advance#1-\wd0%
}%
%%
\def\list#1#2{{%
  %TODO Page breaks are undesired in lists
  % This controll sequence is used to store the content of bullet points.
  \def\lbullet{#1}\def\markierung{}%
%
  %% This controll sequence will create list items and set preopper bullet points. Here it is used to calculate the width of the bullet points and thus the appropriate indentation.
  \def\item##1{%
    \immediate\contextwidth{\bulletwidth}{\lbullet}%
    \rlap{\the\bulletwidth}%FIXME unfortunatly necessary becaus \ifdim won't work otherwise (don't know why please help)
    \ifdim\bulletwidth>\bulletswidth%
      \bulletswidth\bulletwidth%
    \fi%
  }%
  \setbox0\hbox{#2\global\transcendebullets\bulletswidth}\bulletswidth\transcendebullets%
  {%
    %% This controll sequence creates list items and applys bullets to them.
    \def\item##1{%
      %\noindent\lbullet%
      \noindent%
      \lbullet%
      \hfill%
      \vtop{%
        \noindent\strut%
        \advance\hsize-\bulletswidth%
        ##1%
        \strut%
      }%
      \par%
    }%
    #2%
  }%
}}%
