% This script is intended to display numbers as alphabetical characters.
% For example, 0 -> 'a', 1 -> 'b', ..., 25 -> 'z', 26 -> 'aa', 27 -> 'ab', etc.
% It also provides the flexibility to switch to uppercase letters using the TeX \uppercase macro.

\newcount\nextparam % Define a new counter \nextparam
\newcount\digitval % Define a new counter \digitval

\def\alphanumeral#1{{%
  \digitval=#1% Set \digitval to the input number
  \ifnum#1>25 % If the input number is greater than 25 (more than one letter in the sequence)
    \nextparam=#1% Set \nextparam to the input number
    \divide\nextparam by 26 % Divide \nextparam by 26 to get the higher-order digit
    \alphanumeral{\nextparam}% Recursively call \alphanumeral with the higher-order digit
    \multiply\nextparam by 26 % Multiply \nextparam by 26 to get the value back to original scale
    \advance\digitval by -\nextparam % Subtract the scaled \nextparam from \digitval to get the remainder
  \fi
  \advance\digitval by 97 % Add 97 to \digitval to convert it to the ASCII value of lowercase letters
  \char\digitval % Output the character corresponding to the ASCII value in \digitval
}}

%% Example use case:
 %\alphanumeral{0}  % Outputs 'a'
 %\alphanumeral{1}  % Outputs 'b'
 %\alphanumeral{25} % Outputs 'z'
 %\alphanumeral{26} % Outputs 'aa'
 %\alphanumeral{27} % Outputs 'ab'

% To switch to uppercase using the TeX \uppercase macro:
\def\ALPHANUMERAL#1{%
  \uppercase{\expandafter\alphanumeral\expandafter{#1}}%
}

%% Example use case:
 %\ALPHANUMERAL{0}  % Outputs 'A'
 %\ALPHANUMERAL{1}  % Outputs 'B'
 %\ALPHANUMERAL{25} % Outputs 'Z'
 %\ALPHANUMERAL{26} % Outputs 'AA'
 %\ALPHANUMERAL{27} % Outputs 'AB'

 %\bye
